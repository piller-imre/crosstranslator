\Chapter{A nyelv definíciója}

\section{Általános szempontok}

A saját nyelv kialakításánál az volt az elsődleges szempont, hogy ...

E mellett fontos volt még, hogy ...

A szintaxis kialakítása úgy tünt célszerűnek, ...

% QUEST: Milyen szintaktikájú, szemantikájú ~ úgy általában jellegű a nyelv?

% TODO: Láthatóság megvalósítását részletezni.

% TODO: Nyelv fő elemei (típusok, vezérlési szerkezete, ...)

% TODO: Modularitás.

% TODO: Az elterjedtebb nyelvek közül mennyi csak expression és mennyi vegyes?

% TODO: Hibakezelési módok

\section{A célnyelvek áttekintése}

% TODO: Olyasmi kellene, mint a docx-ekben van, csak a többi nyelvvel együtt.

\section{Saját nyeli elemek}

Ebben a fejezetben kellene definiálni a nyelvet.

A nyelv szintaktikája
Vezérlési elemek
Adattípusok és adatszerkezetek
Függvények
Hibakezelés
...

% QUEST: A nyelv expression-öket és/vagy statement-eket tartalmaz?

% TODO: EBNF és/vagy szintaxis diagramos nyelvi definíció.

