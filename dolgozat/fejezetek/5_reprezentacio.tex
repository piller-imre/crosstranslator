\Chapter{Közbülső reprezentáció}

Az a belső objektum szerkezet, amivel egy betöltött programot a keresztfordító magának ábrázol.

Itt kellene részletezni majd azt, hogy ez hogy készült el (konkrét implementációs dolgok).

Ide kerülhetnek példáu olyan osztályok, hogy
\begin{itemize}
\item függvénydefiníció
\item változó definíció
\item for ciklus
\item feltételes elágazás
\item osztály definíció
\item program/modul/package definíció (befoglaló típusnak)
\item blokk/szekvencia típus
\item expression/statement
\end{itemize}

% TODO: Jó sok UML diagram

% TODO: Absztrakt gyár mintát érdemes lesz majd bemutatni a kimeneti programnyelvek implementációjánál.
