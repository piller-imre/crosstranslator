\documentclass[a4paper,12pt]{article}

% Set margins
\usepackage[hmargin=3cm, vmargin=3cm]{geometry}

\frenchspacing

% Language packages
\usepackage[utf8]{inputenc}
\usepackage[T1]{fontenc}
\usepackage[magyar]{babel}

% AMS
\usepackage{amssymb,amsmath}

% Graphic packages
\usepackage{graphicx}

% Colors
\usepackage{color}
\usepackage[usenames,dvipsnames]{xcolor}

% Enumeration
\usepackage{enumitem}

% Links
\usepackage{hyperref}

\linespread{1.2}

\begin{document}

\pagestyle{empty}

\section*{Összegzés}

\textit{Horváth Máté János: Keresztfordító készítése magas szintű programozási nyelvekhez}

\bigskip

A dolgozatban egy új programozási nyelv definiálására került sor. Ehhez a kiinduló pontot azok a már elterjedt, általános programozási nyelvek adták, melyek a keresztfordító szempontjából célnyelvnek tekinthetők. A nyelv definiálása során a dolgozat bemutatta, hogy milyen formában lehet leírni egy programozási nyelvet. A grafikus leírás esetében ez a szintaxis diagramokat jelentette, míg a szöveges leíráshoz a Backus-Naur Forma egy konkrét szintaxisa került felhasználásra.

Egy teljesen általánosan használható keresztfordító létrehozására vonatkozóan a fontos ismérvek részletezésre kerültek a dolgozatban. A keresztfordítás nagyobb fázisait, mint a lexikális elemzést, a közbülső reprezentáció létrehozását, illetve az ebből való célnyelvi visszaírást egy nagy feldolgozási folyamat részének tekintettem. Ezek külön fejezetekben kaptak helyet. Azt igyekeztem hangsúlyozni, hogy minden ilyen feldolgozási lépés függetlennek tekinthető olyan értelemben, hogy a jól definiált be- és kimeneteinek köszönhetően réteges szerkezetű keresztfordítót kapunk, melyek rétegei így külön fejleszthetők, tesztelhetők.

Ahhoz, hogy a létrehozott programozási nyelv népszerű lehessen még rengeteg további munkára van szükség. A dolgozat ennek mindössze az első nagy lépését mutatja be, vagyis felveti a nyelv szükségességét, majd bemutatja a keresztfordítók témakörét, egy új keresztfordító létrehozásának a folyamatát. Mivel a célnyelvek száma nagy (amire a feladat célkitűzését tekintve szükség is volt), az egyes célnyelvek sajátosságainak kezelése további fejlesztéseket igényel. A jövőbeli tervek között szerepel, hogy egy teljesen általánosan használható, széles körben elterjedt nyelv váljon a bemutatott, sajátos programozási nyelvemből.

\end{document}

