\pagestyle{empty} %a címlapon ne legyen semmi=empty, azaz nincs fejléc és lábléc

\begin{flushleft}
\textsc{\bfseries Miskolci Egyetem}\\
Gépészmérnöki és Informatikai Kar\\
Alkalmazott Matematikai Intézeti Tanszék
\end{flushleft}

%A fõiskola logoja
{\large
\begin{center}
\vglue 1truecm
\textbf{\huge\textsc{Szakdolgozat}}\\
\vglue 1truecm
\epsfig{file=cimlap/ME_logo.eps, width=4.8truecm, height=4truecm}\\
\textbf{\textsc{Miskolci Egyetem}}
\end{center}}

\vglue 1.5truecm %függõleges helykihagyás

%A szakdolgozat címe, akár több sorban is
{\LARGE
\begin{center}
\textbf{Keresztfordító készítése magas szintű programozási nyelvekhez}
\end{center}}

\vspace*{2.5truecm}
%A hallgató neve, évfolyam, szak(ok), a konzulens(ek) neve
{\large
\begin{center}
\begin{tabular}{c}
\textbf{Készítette:}\\
Horváth Máté János\\
Mérnökinformatikus BSc
\end{tabular}
\end{center}
\begin{center}
\begin{tabular}{c}
\textbf{Témavezető:}\\
Piller Imre, egyetemi tanársegéd
\end{tabular}
\end{center}}
\vfill
%Keltezés: Hely és év
{\large
\begin{center}
\textbf{\textsc{Miskolc, 2018}}
\end{center}}

\newpage
