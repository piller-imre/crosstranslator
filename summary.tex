\documentclass[a4paper,12pt]{article}

% Set margins
\usepackage[hmargin=3cm, vmargin=3cm]{geometry}

\frenchspacing

% Language packages
\usepackage[utf8]{inputenc}
\usepackage[T1]{fontenc}
\usepackage[magyar]{babel}

% AMS
\usepackage{amssymb,amsmath}

% Graphic packages
\usepackage{graphicx}

% Colors
\usepackage{color}
\usepackage[usenames,dvipsnames]{xcolor}

% Enumeration
\usepackage{enumitem}

% Links
\usepackage{hyperref}

\linespread{1.2}

\begin{document}

\pagestyle{empty}

\section*{Summary}

\textit{Máté János Horváth: Crosscompiler for a high level programming language}

\bigskip

In this work, I have defined a new programming language. It is based on the analysis of the programming languages which were selected as target languages from the aspect of the cross compiler. The methods and tools of the definition of a programming language also has presented. In the case of graphical representation the grammar has described by syntax diagrams, while the textual representation has solved by the Extended Backus-Naur Form.

I have described all of the main aspects of crosscompiler design and implementation. The three important phases of the cross compilation process (namely the lexical analysis, the intermediate representation and the serialization of the target language) has considered from the viewpoint of the overall cross compilation process. These are described in separate chapters. I have tried to emphasize that these steps can be separated and results a layered architecture. It assumes well-defined input and output formats, and makes possible the development and testing of the layers independently.

For making the programming language proper for daily use for the wider audience, a large amount of further development is required. This work only present its first significant step. It reveals the requirement of this language, introduces the topic of cross compilers and the process of cross compiler design and implementation. The number of target languages are large (by the original goal of this work), the handling of language specialities requires more consideration in the future. The long term plan is to form the introduced language to a general purpose, widely used programming language.

\end{document}
