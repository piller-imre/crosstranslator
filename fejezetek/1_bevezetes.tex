\Chapter{Bevezetés}

A dolgozat egy olyan magas szintű programozási nyelv definiálását tűzi ki célul, mely fordítás után más, magas szintű programozási nyelvre fordul le. Ennek segítségével a programkódot nagyon könnyen lehet egyidejűleg több platformra elkészíteni, hiszen a program egy adott nyelven történő megírása után a fordítás során több nyelvre, több platfomra automatikusan elvégezhető lesz a konverzió.

A dolgozat elkészítése során fontos szempont megvizsgálni azon programozási nyelveket, amelyeket a keresztfordítás szempontjából célnyelvnek teküntünk. Nagyon fontos azok szintaktikájának és szemantikájának összehasonlítása, hogy az újonnan definiált nyelv minden szükséges elemet tartalmazzon majd. A dolgozat egyik fontos részét képezik tehát azok a vizsgálatok, melyek a már meglévő nyelveket elemzik különféle szempontok alapján. A dolgozatnak nem célja minden egyes programozási nyelvet teljes egészében bemutatni (erre egy dolgozat keretein belül nem is lenne lehetőség), mindössze azoknak a jellegzetességeknek a kiemelésére kerülhet sor, melyek szükségesek lesznek ahhoz, hogy az új nyelv, és hozzá a keresztfordító definiálható legyen.

A bemutatásra kerülő saját programozási nyelv arra törekszik, hogy minél egyszerűbben használható legyen, a felsorolt célnyelvek közös vonásaira építve egy könnyen tanulható, forráskód szintjén áttekinthető nyelvet eredményezzen. A létrehozásánál a korábbi programozói tapasztalataimra hagyatkoztam, próbáltam a felsorolt célnyelvekben lévő előnyös megoldásokat egy nyelvbe ötvözni.

A keresztfordítás, és úgy általában a nyelvi feldolgozás egy bonyolult problémakör. A dolgozat célja, hogy bemutassa a fő lépéseit a teljes nyelvi feldolgozás folyamatának. Részletesen kitér az egyes lépések céljára, be- és kimeneti adataira, illetve a megvalósításukhoz szükséges alternatív megoldási módokra.

A saját nyelv megadásához a szintaxis diagramokkal történő definíciók, illetve a szöveges formában, EBNF szintaktikának megfelelő nyelvi leírás bizonyult hatékony megoldásnak. Ez egyaránt segíti a nyelvi elemek áttekinthető bemutatását, illetve a hozzájuk készítendő nyelvi feldolgozók szintaxis leírásához is megfelelő alapot adnak.

A keresztfordító elkészítéséhez a Java programozási nyelv tünt megfelelő választásnak. Ezt a Java nyelv hatákonysága és támogatottsága mellett az előzetes programozási tanulmányok is indokolták. A dolgozatban a keresztfordító megvalósításához szükséges elvi háttér bemutatása mellett szerepelnek azok a tervek, amelyek a keresztfordító implementációjához szükségesek. Ez elsősorban a saját definiálású programozási nyelv közbülső reprezentációjának nyelvi elemeihez tartozó osztálydefiníciók formájában taglalja a dolgozat.

A dolgozat végén bemutatásra kerülnek azok a módszerek és eszközök, melyekkel a definiált programozási nyelvhez a keresztfordítót el tudjuk készíteni. A dolgozat kitér az alternatív megoldási lehetőségekre is, illetve részletezi a keresztfordító készítése közben nyert tapasztalatokat.